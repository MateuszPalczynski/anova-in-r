% Options for packages loaded elsewhere
\PassOptionsToPackage{unicode}{hyperref}
\PassOptionsToPackage{hyphens}{url}
%
\documentclass[
]{article}
\title{Projekt}
\author{PałczyńskiMateusz}
\date{4 06 2022}

\usepackage{amsmath,amssymb}
\usepackage{lmodern}
\usepackage{iftex}
\ifPDFTeX
  \usepackage[T1]{fontenc}
  \usepackage[utf8]{inputenc}
  \usepackage{textcomp} % provide euro and other symbols
\else % if luatex or xetex
  \usepackage{unicode-math}
  \defaultfontfeatures{Scale=MatchLowercase}
  \defaultfontfeatures[\rmfamily]{Ligatures=TeX,Scale=1}
\fi
% Use upquote if available, for straight quotes in verbatim environments
\IfFileExists{upquote.sty}{\usepackage{upquote}}{}
\IfFileExists{microtype.sty}{% use microtype if available
  \usepackage[]{microtype}
  \UseMicrotypeSet[protrusion]{basicmath} % disable protrusion for tt fonts
}{}
\makeatletter
\@ifundefined{KOMAClassName}{% if non-KOMA class
  \IfFileExists{parskip.sty}{%
    \usepackage{parskip}
  }{% else
    \setlength{\parindent}{0pt}
    \setlength{\parskip}{6pt plus 2pt minus 1pt}}
}{% if KOMA class
  \KOMAoptions{parskip=half}}
\makeatother
\usepackage{xcolor}
\IfFileExists{xurl.sty}{\usepackage{xurl}}{} % add URL line breaks if available
\IfFileExists{bookmark.sty}{\usepackage{bookmark}}{\usepackage{hyperref}}
\hypersetup{
  pdftitle={Projekt},
  pdfauthor={PałczyńskiMateusz},
  hidelinks,
  pdfcreator={LaTeX via pandoc}}
\urlstyle{same} % disable monospaced font for URLs
\usepackage[margin=1in]{geometry}
\usepackage{color}
\usepackage{fancyvrb}
\newcommand{\VerbBar}{|}
\newcommand{\VERB}{\Verb[commandchars=\\\{\}]}
\DefineVerbatimEnvironment{Highlighting}{Verbatim}{commandchars=\\\{\}}
% Add ',fontsize=\small' for more characters per line
\usepackage{framed}
\definecolor{shadecolor}{RGB}{248,248,248}
\newenvironment{Shaded}{\begin{snugshade}}{\end{snugshade}}
\newcommand{\AlertTok}[1]{\textcolor[rgb]{0.94,0.16,0.16}{#1}}
\newcommand{\AnnotationTok}[1]{\textcolor[rgb]{0.56,0.35,0.01}{\textbf{\textit{#1}}}}
\newcommand{\AttributeTok}[1]{\textcolor[rgb]{0.77,0.63,0.00}{#1}}
\newcommand{\BaseNTok}[1]{\textcolor[rgb]{0.00,0.00,0.81}{#1}}
\newcommand{\BuiltInTok}[1]{#1}
\newcommand{\CharTok}[1]{\textcolor[rgb]{0.31,0.60,0.02}{#1}}
\newcommand{\CommentTok}[1]{\textcolor[rgb]{0.56,0.35,0.01}{\textit{#1}}}
\newcommand{\CommentVarTok}[1]{\textcolor[rgb]{0.56,0.35,0.01}{\textbf{\textit{#1}}}}
\newcommand{\ConstantTok}[1]{\textcolor[rgb]{0.00,0.00,0.00}{#1}}
\newcommand{\ControlFlowTok}[1]{\textcolor[rgb]{0.13,0.29,0.53}{\textbf{#1}}}
\newcommand{\DataTypeTok}[1]{\textcolor[rgb]{0.13,0.29,0.53}{#1}}
\newcommand{\DecValTok}[1]{\textcolor[rgb]{0.00,0.00,0.81}{#1}}
\newcommand{\DocumentationTok}[1]{\textcolor[rgb]{0.56,0.35,0.01}{\textbf{\textit{#1}}}}
\newcommand{\ErrorTok}[1]{\textcolor[rgb]{0.64,0.00,0.00}{\textbf{#1}}}
\newcommand{\ExtensionTok}[1]{#1}
\newcommand{\FloatTok}[1]{\textcolor[rgb]{0.00,0.00,0.81}{#1}}
\newcommand{\FunctionTok}[1]{\textcolor[rgb]{0.00,0.00,0.00}{#1}}
\newcommand{\ImportTok}[1]{#1}
\newcommand{\InformationTok}[1]{\textcolor[rgb]{0.56,0.35,0.01}{\textbf{\textit{#1}}}}
\newcommand{\KeywordTok}[1]{\textcolor[rgb]{0.13,0.29,0.53}{\textbf{#1}}}
\newcommand{\NormalTok}[1]{#1}
\newcommand{\OperatorTok}[1]{\textcolor[rgb]{0.81,0.36,0.00}{\textbf{#1}}}
\newcommand{\OtherTok}[1]{\textcolor[rgb]{0.56,0.35,0.01}{#1}}
\newcommand{\PreprocessorTok}[1]{\textcolor[rgb]{0.56,0.35,0.01}{\textit{#1}}}
\newcommand{\RegionMarkerTok}[1]{#1}
\newcommand{\SpecialCharTok}[1]{\textcolor[rgb]{0.00,0.00,0.00}{#1}}
\newcommand{\SpecialStringTok}[1]{\textcolor[rgb]{0.31,0.60,0.02}{#1}}
\newcommand{\StringTok}[1]{\textcolor[rgb]{0.31,0.60,0.02}{#1}}
\newcommand{\VariableTok}[1]{\textcolor[rgb]{0.00,0.00,0.00}{#1}}
\newcommand{\VerbatimStringTok}[1]{\textcolor[rgb]{0.31,0.60,0.02}{#1}}
\newcommand{\WarningTok}[1]{\textcolor[rgb]{0.56,0.35,0.01}{\textbf{\textit{#1}}}}
\usepackage{graphicx}
\makeatletter
\def\maxwidth{\ifdim\Gin@nat@width>\linewidth\linewidth\else\Gin@nat@width\fi}
\def\maxheight{\ifdim\Gin@nat@height>\textheight\textheight\else\Gin@nat@height\fi}
\makeatother
% Scale images if necessary, so that they will not overflow the page
% margins by default, and it is still possible to overwrite the defaults
% using explicit options in \includegraphics[width, height, ...]{}
\setkeys{Gin}{width=\maxwidth,height=\maxheight,keepaspectratio}
% Set default figure placement to htbp
\makeatletter
\def\fps@figure{htbp}
\makeatother
\setlength{\emergencystretch}{3em} % prevent overfull lines
\providecommand{\tightlist}{%
  \setlength{\itemsep}{0pt}\setlength{\parskip}{0pt}}
\setcounter{secnumdepth}{-\maxdimen} % remove section numbering
\ifLuaTeX
  \usepackage{selnolig}  % disable illegal ligatures
\fi

\begin{document}
\maketitle

\hypertarget{ux15brednia-jakoux15bux107-batonuxf3w-czekoladowych-w-zaleux17cnoux15bci-od-kraju-pochodzenia}{%
\section{Średnia jakość batonów czekoladowych w zależności od kraju
pochodzenia}\label{ux15brednia-jakoux15bux107-batonuxf3w-czekoladowych-w-zaleux17cnoux15bci-od-kraju-pochodzenia}}

\hypertarget{spis-treux15bci}{%
\subsection{Spis treści:}\label{spis-treux15bci}}

\hypertarget{ocena-jakoux15bci-batonikuxf3w-przez-respondentuxf3w.}{%
\subsection{1.Ocena jakości batoników przez
respondentów.}\label{ocena-jakoux15bci-batonikuxf3w-przez-respondentuxf3w.}}

\hypertarget{wczytywanie-danych}{%
\subsection{2.Wczytywanie danych}\label{wczytywanie-danych}}

\hypertarget{wstux119pna-analiza-danych}{%
\subsection{3.Wstępna analiza danych}\label{wstux119pna-analiza-danych}}

\hypertarget{postawienie-problemu}{%
\subsection{4.Postawienie problemu}\label{postawienie-problemu}}

\hypertarget{sprawdzenie-zaux142oux17ceux144-anova}{%
\subsection{5.Sprawdzenie założeń
ANOVA}\label{sprawdzenie-zaux142oux17ceux144-anova}}

\hypertarget{test-anova}{%
\subsection{6.Test ANOVA}\label{test-anova}}

\hypertarget{test-post-hoc}{%
\subsection{7.Test post-hoc}\label{test-post-hoc}}

\hypertarget{wnioski}{%
\subsection{8.Wnioski}\label{wnioski}}

\hypertarget{ocena-jakoux15bci-batonikuxf3w-przez-respondentuxf3w.-1}{%
\subsection{1. Ocena jakości batoników przez
respondentów.}\label{ocena-jakoux15bci-batonikuxf3w-przez-respondentuxf3w.-1}}

Dane dotyczą ocen batonów z odpowiednich krajów. W ich skład wchodzą:
nazwa firmy, wskaźnik referencyjny, rok zebrania danych, procent
zawartości kakao, ocena i rodzaj nasion kakao oraz kraju pochodzenia
nasion. Batony oceniane są w skali 1 --- 5 przez 5 losowych grup około
15 osób. Każda z tych grup dotyczyła oceny w innym kraju.

\hypertarget{wczytywanie-danych-1}{%
\subsection{2. Wczytywanie danych}\label{wczytywanie-danych-1}}

\begin{Shaded}
\begin{Highlighting}[]
\NormalTok{dane1}\OtherTok{\textless{}{-}} \FunctionTok{read.csv}\NormalTok{(}\StringTok{\textquotesingle{}flavors\_of\_cacao.csv\textquotesingle{}}\NormalTok{, }\AttributeTok{head =} \ConstantTok{TRUE}\NormalTok{)}
\CommentTok{\#w celach badawcych bierzemy miarodajną próbkę z bazy danych}
\NormalTok{dane }\OtherTok{\textless{}{-}}\NormalTok{ dane1[}\FunctionTok{c}\NormalTok{(}\DecValTok{10}\SpecialCharTok{:}\DecValTok{21}\NormalTok{, }\DecValTok{45}\SpecialCharTok{:}\DecValTok{59}\NormalTok{, }\DecValTok{76}\SpecialCharTok{:}\DecValTok{88}\NormalTok{,}\DecValTok{122}\SpecialCharTok{:}\DecValTok{136}\NormalTok{, }\DecValTok{448}\SpecialCharTok{:}\DecValTok{462}\NormalTok{), ]}
\FunctionTok{head}\NormalTok{(dane)}
\end{Highlighting}
\end{Shaded}

\begin{verbatim}
##    CompanyÂ...Maker.if.known. Specific.Bean.Origin.or.Bar.Name  REF Review.Date
## 10                   A. Morin                          Pablino 1319        2014
## 11                   A. Morin                           Panama 1011        2013
## 12                   A. Morin                       Madagascar 1011        2013
## 13                   A. Morin                           Brazil 1011        2013
## 14                   A. Morin                         Equateur 1011        2013
## 15                   A. Morin                         Colombie 1015        2013
##    Cocoa.Percent Company.Location Rating Bean.Type Broad.Bean.Origin
## 10           70%           France   4.00                       Peru
## 11           70%           France   2.75                     Panama
## 12           70%           France   3.00   Criollo        Madagascar
## 13           70%           France   3.25                     Brazil
## 14           70%           France   3.75                    Ecuador
## 15           70%           France   2.75                   Colombia
\end{verbatim}

\begin{Shaded}
\begin{Highlighting}[]
\NormalTok{lokalizacje }\OtherTok{\textless{}{-}}\NormalTok{ dane }\SpecialCharTok{$}\NormalTok{ Company.Location }
\NormalTok{oceny }\OtherTok{\textless{}{-}}\NormalTok{ dane }\SpecialCharTok{$}\NormalTok{ Rating}
\NormalTok{lok\_oceny }\OtherTok{\textless{}{-}} \FunctionTok{factor}\NormalTok{ (lokalizacje,}
             \AttributeTok{levels =} \FunctionTok{c}\NormalTok{(}\StringTok{"France"}\NormalTok{, }\StringTok{"U.S.A."}\NormalTok{, }\StringTok{"Italy"}\NormalTok{, }\StringTok{"U.K."}\NormalTok{, }\StringTok{"Germany"}\NormalTok{))}
\end{Highlighting}
\end{Shaded}

\hypertarget{wstux119pna-analiza-danych-1}{%
\subsection{3. Wstępna analiza
danych}\label{wstux119pna-analiza-danych-1}}

\begin{Shaded}
\begin{Highlighting}[]
\FunctionTok{summary}\NormalTok{(oceny)}
\end{Highlighting}
\end{Shaded}

\begin{verbatim}
##    Min. 1st Qu.  Median    Mean 3rd Qu.    Max. 
##   1.500   3.000   3.250   3.279   3.750   5.000
\end{verbatim}

\begin{Shaded}
\begin{Highlighting}[]
\NormalTok{średnie }\OtherTok{\textless{}{-}} \FunctionTok{by}\NormalTok{(oceny ,lokalizacje ,mean)}
\NormalTok{średnie}
\end{Highlighting}
\end{Shaded}

\begin{verbatim}
## lokalizacje: France
## [1] 3.375
## ------------------------------------------------------------ 
## lokalizacje: Germany
## [1] 3.15
## ------------------------------------------------------------ 
## lokalizacje: Italy
## [1] 3.846154
## ------------------------------------------------------------ 
## lokalizacje: U.K.
## [1] 3.033333
## ------------------------------------------------------------ 
## lokalizacje: U.S.A.
## [1] 3.083333
\end{verbatim}

\begin{Shaded}
\begin{Highlighting}[]
\FunctionTok{boxplot}\NormalTok{(oceny }\SpecialCharTok{\textasciitilde{}}\NormalTok{ lokalizacje, }\AttributeTok{range =} \DecValTok{0}\NormalTok{)}
\FunctionTok{points}\NormalTok{(średnie, }\AttributeTok{col =} \StringTok{"purple"}\NormalTok{, }\AttributeTok{pch =} \DecValTok{16}\NormalTok{)}
\end{Highlighting}
\end{Shaded}

\includegraphics{Projekt-1-1_files/figure-latex/unnamed-chunk-4-1.pdf}
\#\# 4. Postawienie problemu Stawiamy hipotezę, że średnie oceny batonów
ze wszystkich krajów są sobie równ przeciw hipotezie alternatywnej -
istnieją takie dwie lokazizacje, dla których średnie są różne.

\hypertarget{sprawdzenie-zaux142oux17ceux144-anova-1}{%
\subsection{5. Sprawdzenie założeń
ANOVA}\label{sprawdzenie-zaux142oux17ceux144-anova-1}}

I.Dane te zostały pozyskane w sposób losowy - spełnione. II.Pochodzą z
rozkładu normalnego - narysujmy histogramy dla poszczególnych populacji.
Naniesiemy na wykresy odpowiadające kolejnym ocenom gęstości.

\begin{Shaded}
\begin{Highlighting}[]
\NormalTok{France }\OtherTok{=} \FunctionTok{subset}\NormalTok{(dane, }\AttributeTok{subset =}\NormalTok{ (Company.Location }\SpecialCharTok{==} \StringTok{"France"}\NormalTok{))}
\NormalTok{U.S.A. }\OtherTok{=} \FunctionTok{subset}\NormalTok{(dane, }\AttributeTok{subset =}\NormalTok{ (Company.Location }\SpecialCharTok{==} \StringTok{"U.S.A."}\NormalTok{))}
\NormalTok{U.K. }\OtherTok{=} \FunctionTok{subset}\NormalTok{(dane, }\AttributeTok{subset =}\NormalTok{ (Company.Location }\SpecialCharTok{==} \StringTok{"U.K."}\NormalTok{))}
\NormalTok{Italy }\OtherTok{=} \FunctionTok{subset}\NormalTok{(dane, }\AttributeTok{subset =}\NormalTok{ (Company.Location }\SpecialCharTok{==} \StringTok{"Italy"}\NormalTok{))}
\NormalTok{Germany }\OtherTok{=} \FunctionTok{subset}\NormalTok{(dane, }\AttributeTok{subset =}\NormalTok{ (Company.Location }\SpecialCharTok{==} \StringTok{"Germany"}\NormalTok{))}

\FunctionTok{par}\NormalTok{(}\AttributeTok{mfrow=}\FunctionTok{c}\NormalTok{(}\DecValTok{2}\NormalTok{,}\DecValTok{3}\NormalTok{))}
\FunctionTok{hist}\NormalTok{(France[,}\DecValTok{7}\NormalTok{],}\AttributeTok{las =} \DecValTok{1}\NormalTok{, }\AttributeTok{main =} \StringTok{"France histogram"}\NormalTok{,}\AttributeTok{prob =} \ConstantTok{TRUE}\NormalTok{,}
                      \AttributeTok{col =} \StringTok{"orange1"}\NormalTok{,}
                      \AttributeTok{lwd =} \DecValTok{2}\NormalTok{,}
                      \AttributeTok{xlab =} \StringTok{"Oceny {-} Francja"}\NormalTok{,}
                      \AttributeTok{ylim=}\FunctionTok{c}\NormalTok{(}\DecValTok{0}\NormalTok{,}\FloatTok{1.5}\NormalTok{))}
\FunctionTok{lines}\NormalTok{(}\FunctionTok{density}\NormalTok{(France[,}\DecValTok{7}\NormalTok{]),}\AttributeTok{lwd =} \DecValTok{2}\NormalTok{)}

\FunctionTok{hist}\NormalTok{(U.S.A.[,}\DecValTok{7}\NormalTok{],}\AttributeTok{las =} \DecValTok{1}\NormalTok{, }\AttributeTok{main =} \StringTok{"U.S.A. histogram"}\NormalTok{,}\AttributeTok{prob =} \ConstantTok{TRUE}\NormalTok{,}
                      \AttributeTok{col =} \StringTok{"red"}\NormalTok{,}
                      \AttributeTok{lwd =} \DecValTok{2}\NormalTok{,}
                      \AttributeTok{xlab =} \StringTok{"oceny {-} U.S.A."}\NormalTok{,}
                      \AttributeTok{ylim=}\FunctionTok{c}\NormalTok{(}\DecValTok{0}\NormalTok{,}\FloatTok{1.5}\NormalTok{))}
\FunctionTok{lines}\NormalTok{(}\FunctionTok{density}\NormalTok{(U.S.A.[,}\DecValTok{7}\NormalTok{]),}\AttributeTok{lwd =} \DecValTok{2}\NormalTok{)}

\FunctionTok{hist}\NormalTok{(U.K.[,}\DecValTok{7}\NormalTok{],}\AttributeTok{las =} \DecValTok{1}\NormalTok{, }\AttributeTok{main =} \StringTok{"U.K. histogram"}\NormalTok{,}\AttributeTok{prob =} \ConstantTok{TRUE}\NormalTok{,}
                      \AttributeTok{col =} \StringTok{"blue"}\NormalTok{,}
                      \AttributeTok{lwd =} \DecValTok{2}\NormalTok{,}
                      \AttributeTok{xlab =} \StringTok{"oceny {-} U.K."}\NormalTok{,}
                      \AttributeTok{ylim=}\FunctionTok{c}\NormalTok{(}\DecValTok{0}\NormalTok{,}\FloatTok{1.5}\NormalTok{))}
\FunctionTok{lines}\NormalTok{(}\FunctionTok{density}\NormalTok{(U.K.[,}\DecValTok{7}\NormalTok{]),}\AttributeTok{lwd =} \DecValTok{2}\NormalTok{)}

\FunctionTok{hist}\NormalTok{(Germany[,}\DecValTok{7}\NormalTok{],}\AttributeTok{las =} \DecValTok{1}\NormalTok{, }\AttributeTok{main =} \StringTok{"Germany histogram"}\NormalTok{,}\AttributeTok{prob =} \ConstantTok{TRUE}\NormalTok{,}
                      \AttributeTok{col =} \StringTok{"purple"}\NormalTok{,}
                      \AttributeTok{lwd =} \DecValTok{2}\NormalTok{,}
                      \AttributeTok{xlab =} \StringTok{"oceny {-} Niemcy"}\NormalTok{,}
                      \AttributeTok{ylim=}\FunctionTok{c}\NormalTok{(}\DecValTok{0}\NormalTok{,}\FloatTok{1.5}\NormalTok{))}
\FunctionTok{lines}\NormalTok{(}\FunctionTok{density}\NormalTok{(Germany[,}\DecValTok{7}\NormalTok{]),}\AttributeTok{lwd =} \DecValTok{2}\NormalTok{)}

\FunctionTok{hist}\NormalTok{(Italy[,}\DecValTok{7}\NormalTok{],}\AttributeTok{las =} \DecValTok{1}\NormalTok{, }\AttributeTok{main =} \StringTok{"Italy histogram"}\NormalTok{,}\AttributeTok{prob =} \ConstantTok{TRUE}\NormalTok{,}
                      \AttributeTok{col =} \StringTok{"green"}\NormalTok{,}
                      \AttributeTok{lwd =} \DecValTok{2}\NormalTok{,}
                      \AttributeTok{xlab =} \StringTok{"oceny {-} Włochy"}\NormalTok{,}
                      \AttributeTok{ylim=}\FunctionTok{c}\NormalTok{(}\DecValTok{0}\NormalTok{,}\FloatTok{1.5}\NormalTok{))}
\FunctionTok{lines}\NormalTok{(}\FunctionTok{density}\NormalTok{(Italy[,}\DecValTok{7}\NormalTok{]),}\AttributeTok{lwd =} \DecValTok{2}\NormalTok{)}
\end{Highlighting}
\end{Shaded}

\includegraphics{Projekt-1-1_files/figure-latex/unnamed-chunk-5-1.pdf}
Oraz sprawdzimy czy próbka pochodzi z rozkładu normalnego za pomocą
testu Shapiro-Wilka.

\begin{Shaded}
\begin{Highlighting}[]
\FunctionTok{shapiro.test}\NormalTok{(oceny [lok\_oceny }\SpecialCharTok{==} \StringTok{"France"}\NormalTok{])}
\end{Highlighting}
\end{Shaded}

\begin{verbatim}
## 
##  Shapiro-Wilk normality test
## 
## data:  oceny[lok_oceny == "France"]
## W = 0.89475, p-value = 0.1357
\end{verbatim}

\begin{Shaded}
\begin{Highlighting}[]
\FunctionTok{shapiro.test}\NormalTok{(oceny [lok\_oceny }\SpecialCharTok{==} \StringTok{"U.S.A."}\NormalTok{])}
\end{Highlighting}
\end{Shaded}

\begin{verbatim}
## 
##  Shapiro-Wilk normality test
## 
## data:  oceny[lok_oceny == "U.S.A."]
## W = 0.87719, p-value = 0.04308
\end{verbatim}

\begin{Shaded}
\begin{Highlighting}[]
\FunctionTok{shapiro.test}\NormalTok{(oceny [lok\_oceny }\SpecialCharTok{==} \StringTok{"Germany"}\NormalTok{])}
\end{Highlighting}
\end{Shaded}

\begin{verbatim}
## 
##  Shapiro-Wilk normality test
## 
## data:  oceny[lok_oceny == "Germany"]
## W = 0.79404, p-value = 0.003086
\end{verbatim}

\begin{Shaded}
\begin{Highlighting}[]
\FunctionTok{shapiro.test}\NormalTok{(oceny [lok\_oceny }\SpecialCharTok{==} \StringTok{"Italy"}\NormalTok{])}
\end{Highlighting}
\end{Shaded}

\begin{verbatim}
## 
##  Shapiro-Wilk normality test
## 
## data:  oceny[lok_oceny == "Italy"]
## W = 0.88576, p-value = 0.08539
\end{verbatim}

\begin{Shaded}
\begin{Highlighting}[]
\FunctionTok{shapiro.test}\NormalTok{(oceny [lok\_oceny }\SpecialCharTok{==} \StringTok{"U.K."}\NormalTok{])}
\end{Highlighting}
\end{Shaded}

\begin{verbatim}
## 
##  Shapiro-Wilk normality test
## 
## data:  oceny[lok_oceny == "U.K."]
## W = 0.92209, p-value = 0.2073
\end{verbatim}

Dane dla Niemiec i USA nie spełniają naszych założeń o tym, że są z
rozkładu normalnego. W celach badawczych przeprowadzimy jednak naszą
analizę.

III.Jednorodność wariancji zmiennych losowych.

\begin{Shaded}
\begin{Highlighting}[]
\FunctionTok{by}\NormalTok{(oceny ,lokalizacje , var )}
\end{Highlighting}
\end{Shaded}

\begin{verbatim}
## lokalizacje: France
## [1] 0.2215909
## ------------------------------------------------------------ 
## lokalizacje: Germany
## [1] 0.3107143
## ------------------------------------------------------------ 
## lokalizacje: Italy
## [1] 0.380609
## ------------------------------------------------------------ 
## lokalizacje: U.K.
## [1] 0.4363095
## ------------------------------------------------------------ 
## lokalizacje: U.S.A.
## [1] 0.2470238
\end{verbatim}

Wariancje poszczególnych wyników z lokalizacji nie odbiegają znacznie od
siebie. Jednak dla upewnienia się przeprowadzimy test Barletta.

\begin{Shaded}
\begin{Highlighting}[]
\FunctionTok{bartlett.test}\NormalTok{(oceny }\SpecialCharTok{\textasciitilde{}}\NormalTok{ lokalizacje, }\AttributeTok{data =}\NormalTok{ dane)}
\end{Highlighting}
\end{Shaded}

\begin{verbatim}
## 
##  Bartlett test of homogeneity of variances
## 
## data:  oceny by lokalizacje
## Bartlett's K-squared = 1.9736, df = 4, p-value = 0.7406
\end{verbatim}

Test potwierdził nasze przypuszczenie. \#\# 6. Test ANOVA Możemy teraz
przejść do naszego finalnego testu ANOVA

\begin{Shaded}
\begin{Highlighting}[]
\FunctionTok{anova}\NormalTok{ (}\FunctionTok{lm}\NormalTok{(oceny }\SpecialCharTok{\textasciitilde{}}\NormalTok{ lokalizacje))}
\end{Highlighting}
\end{Shaded}

\begin{verbatim}
## Analysis of Variance Table
## 
## Response: oceny
##             Df  Sum Sq Mean Sq F value   Pr(>F)   
## lokalizacje  4  6.0214 1.50535  4.6769 0.002223 **
## Residuals   65 20.9215 0.32187                    
## ---
## Signif. codes:  0 '***' 0.001 '**' 0.01 '*' 0.05 '.' 0.1 ' ' 1
\end{verbatim}

Odrzucamy hipotezę zerową, tzn. istnieje taka para krajów, z których
średnie ocen batonów są od siebie różne. Najmniejszym poziomem
istotności z tutaj wymienionych dla którego odrzucamy hipotezę zerową
jest 0.001.

\hypertarget{test-post-hoc-1}{%
\subsection{7. Test post-hoc}\label{test-post-hoc-1}}

Przeprowadźmy test post-hoc (na poziomie ufności 95\%) --- Test Tukeya,
aby wzbogacić analizę o dodatkowe --- dokładniejsze wnioski; średnie
ocen z których krajów są parami różne.

\begin{Shaded}
\begin{Highlighting}[]
\FunctionTok{TukeyHSD}\NormalTok{ (}\FunctionTok{aov}\NormalTok{(oceny }\SpecialCharTok{\textasciitilde{}}\NormalTok{ lok\_oceny ), }\AttributeTok{conf.level =} \FloatTok{0.95}\NormalTok{)}
\end{Highlighting}
\end{Shaded}

\begin{verbatim}
##   Tukey multiple comparisons of means
##     95% family-wise confidence level
## 
## Fit: aov(formula = oceny ~ lok_oceny)
## 
## $lok_oceny
##                       diff        lwr         upr     p adj
## U.S.A.-France  -0.29166667 -0.9081843  0.32485101 0.6753150
## Italy-France    0.47115385 -0.1660929  1.10840057 0.2435613
## U.K.-France    -0.34166667 -0.9581843  0.27485101 0.5312527
## Germany-France -0.22500000 -0.8415177  0.39151768 0.8433866
## Italy-U.S.A.    0.76282051  0.1596201  1.36602091 0.0063099
## U.K.-U.S.A.    -0.05000000 -0.6312584  0.53125844 0.9992300
## Germany-U.S.A.  0.06666667 -0.5145918  0.64792511 0.9976200
## U.K.-Italy     -0.81282051 -1.4160209 -0.20962012 0.0030572
## Germany-Italy  -0.69615385 -1.2993542 -0.09295345 0.0157230
## Germany-U.K.    0.11666667 -0.4645918  0.69792511 0.9798777
\end{verbatim}

\hypertarget{wnioski-1}{%
\subsection{8. Wnioski}\label{wnioski-1}}

Zauważmy, że średnie są różne jedynie dla relacji: Włochy-U.S.A.,
Włochy-Wielka Brytania, Włochy-Niemcy. Możemy więc stwierdzić, że
średnia jakość batoników we Włoszech jest widocznie lepsza niżeli w
Stanach Zjednoczonych, Niemczech czy na Wyspach Brytyjskich. Nie mamy
jednak wiedzy by uznać wyższość batoników z Włoch ponad tych
pochodzących z Francji.

\end{document}
